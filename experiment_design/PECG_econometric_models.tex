%% LyX 2.2.1 created this file.  For more info, see http://www.lyx.org/.
%% Do not edit unless you really know what you are doing.
\RequirePackage{fixltx2e}
\documentclass[12pt,english,marginparwidth=8cm, marginparsep=3mm]{article}
\usepackage{lmodern}
\renewcommand{\sfdefault}{lmss}
\renewcommand{\ttdefault}{lmtt}
\usepackage[T1]{fontenc}
\usepackage[latin9]{inputenc}
\usepackage[a4paper]{geometry}
\geometry{verbose,tmargin=1.5cm,bmargin=1.5cm,lmargin=1.5cm,rmargin=1.5cm}
\pagestyle{plain}
\usepackage{color}
\usepackage{babel}
\usepackage{enumitem}
\usepackage{amsmath}
\usepackage{amsthm}
\usepackage{setspace}
\usepackage[authoryear]{natbib}
\PassOptionsToPackage{normalem}{ulem}
\usepackage{ulem}
\onehalfspacing
\usepackage[unicode=true,pdfusetitle,
 bookmarks=true,bookmarksnumbered=false,bookmarksopen=false,
 breaklinks=true,pdfborder={0 0 0},pdfborderstyle={},backref=false,colorlinks=true]
 {hyperref}
\usepackage{breakurl}

\makeatletter

%%%%%%%%%%%%%%%%%%%%%%%%%%%%%% LyX specific LaTeX commands.
%% Because html converters don't know tabularnewline
\providecommand{\tabularnewline}{\\}

%%%%%%%%%%%%%%%%%%%%%%%%%%%%%% Textclass specific LaTeX commands.
\numberwithin{equation}{section}
\numberwithin{figure}{section}
\numberwithin{table}{section}
\newlength{\lyxlabelwidth}      % auxiliary length

\@ifundefined{date}{}{\date{}}
%%%%%%%%%%%%%%%%%%%%%%%%%%%%%% User specified LaTeX commands.
\usepackage{graphicx}
\usepackage{hyperref}
  \hypersetup{colorlinks=true,citecolor=blue,linkcolor=blue}

%\setlength{\marginparwidth}{5cm}

\makeatother

\begin{document}

\subsection{Background}

We are running an experiment on peer/social influences on donations
in an online fundraising setting. The aim of this study is to assess
the impact larger-than-average (and smaller-than-average) additional
donation(s) to a fundraiser\textquoteright s page can have on subsequent
donations to that page. This is of academic and practical interest,
including for those in the Effective Altruism (EA) community who are
considering whether to donate independently or via friends and strangers'
fundraisers.

We are running a field experiment/randomized controlled trial in a
natural environment. Essentially, we observe all newly created fundraising
pages on the JustGiving platform meeting our criteria. We will assign
these to Treatment and Control groups in a way specified in our pre-registration
(blocked randomization, with the randomization component {[}possibly{]}
based on an ex-post observable device). For the Treatment pages we
will make a single anonymous donation on this page (from our personal
funds), with gift-aid, of a pre-specified size {[}relative to prior
contributions{]}. The unit of randomization is the fundraising page.
The 'participants' are all subsequent visitors to the fundraising
page, including the page founder (who may take subsequent actions
to promote the page).

We will continue to run the above until we have attained sufficient
statistical power (as defined in our preregistration) or until we
run out of funds (also defined). We will observe and collect data
from these sites until they are closed, or until the point where each
of these are likely to have obtained all or most of their donations
(also defined).

We will analyze the impact of our treatment on several outcomes, principally
focusing on it's impact on {*}total{*} contributions (excluding our
treatment contribution), but also analyzing the time-distribution
of these and the nature of the peer-effects, and examining heterogeneity
in was specified in our pre-analysis plan (PaP).

\section{\label{sec:EMs}Econometric Models}

We exogenously assign whether we add an anonymous small donation $(D=s)$,
an anonymous large donation $(D=l)$, or no donation $(D=0)$ to a
page (the proposed exogenous assignment procedure is described in
section {[}ref{]} below).

\emph{A simple model of a single treatment is proposed below, by Eirini
Tatsi.}

Suppose our sample consists of treated and control pages.

We model the donation of subsequent donors, as well as the total amount
raised by the page.

A standard model, following \citet{Bursztynetal2014}, for donor $i=1,...,n$
is:

\begin{equation}
y_{i}=\alpha+I_{i}\beta+x_{i}\gamma+\varepsilon_{i},\label{eq:donor_i}
\end{equation}
in which $y$ denotes \textendash{} for instance \textendash{} the
amount donated by individual $i=1,...,n$; $\alpha$ is a constant
term, $I_{i}$ is an indicator variable taking the value $1$ if the
individual belongs to the treatment group and $0$ otherwise, $x$
are donors' characteristics while $\varepsilon_{i}$ is an independent
and identically distributed error term.{*}\footnote{DR: We may later allow for time and ordering effects... donations
may come in waves instigated by external unobservable factors.} The outcome variable $y$ is necessarily the amount the individual
donated or how much they donated compared to a benchmark amount because
we only observe donations and not the decision to donate (as in \citealp{Bursztynetal2014}).
Scalar parameter $\beta$ is the parameter of interest as it identifies
the one of the experimental treatment effects of interest{*}\footnote{DR: We need to define this treatment effect more carefully. Not sure
we can identify the impact 'conditional on a positive donation. Also
for policy purposes we are also interested in other outcomes, particularly
the total amount raised, which will be affected through several channels,
including the arrival rate, the rate of donating a positive amount
after arriving, and by indirect peer effects.} Matrix $x$ might include the donor's gender (if this is identifiable
using the names)\textendash{} or if the donor has a social-media connection
to the person who started the charity page{*}\footnote{DR: do we observe the social media connection?}
(since according to \citealp{Smithetal2013} those who donate are
usually friends, family or colleagues of the person who started the
charity page); it can also include an indicator variable taking value
$1$ if the donor was anonymous.{*}\footnote{DR: Careful-{}-the decision to be anonymous is likely to be affected
by the treatment here}

In reality, donors to the same page might share common unobserved
characteristics, e.g., preferences to contribute to an animal-rights
cause or an enviromental cause. Therefore, the econometric model in
equation \ref{eq:donor_i} should include page $p=1,...,P$ fixed
effects:{*}\footnote{DR: Calling them 'fixed efects' may confuse people as these will be
exogenous to the treatments and we aren't planning to use the standard
\textquotedbl{}fixed effects\textquotedbl{} estimator.}

\begin{equation}
y_{i}=\alpha_{p}+I_{i}\beta+x_{i}\gamma+\varepsilon_{i}.\label{eq:donor_i_page_fe}
\end{equation}
For the same reason, in estimation standard errors could be clustered
at the page level. Writing the model in \ref{eq:donor_i_page_fe}
for the $n$ observations of the $P$ pages, we get:

\begin{equation}
Y_{n}=\alpha_{n}+I_{n}\beta+X_{n}\gamma+\varepsilon_{n},\label{eq:donor_n_page_fe}
\end{equation}
in which $Y_{n}$ denotes the $n\times1$ vector of outcomes, $\alpha_{n}=\left[\alpha_{1}\cdots\alpha_{1}\cdots\cdots\alpha_{P}\cdots\alpha_{P}\right]$,
$I_{n}$ denotes the $n\times1$ vector of treatment dummies, $X_{n}$
is the $n\times k$ matrix of explanatory variables and $\varepsilon_{n}$
is the $n\times1$ vector of i.i.d. error terms. To allow for errors
of of the same page to be correlated,\footnote{According to \citealp{Barriosetal2012}, if ``the covariate of interest
is randomly assigned at the cluster level'' \textendash{} herein
page \textendash{} calculating standard errors beyond clustering might
be redundant.} we can define:

\begin{equation}
Y_{n}=\alpha_{n}+I_{n}\beta+X_{n}\gamma+u_{n},\label{eq:donor_n_page_fe_sem}
\end{equation}

\begin{equation}
u_{n}=\rho W_{n}u_{n}+\varepsilon_{n},
\end{equation}
in which $W_{n}$ is a block-diagonal weights matrix with non-zero
elements if two donations were made at the same page.

When visiting the charity webpage, one can see immediately the previous
five donations. Assuming for expositional purposes that our assigned
donation treatment is the \emph{third}, then the information visible
is as follows:\\

{\scriptsize{}}%
\begin{tabular}{|c|c|c|c|c|c|c|}
\hline
{\scriptsize{}$t=1$} & {\scriptsize{}$t=2$} & {\scriptsize{}$t=3$} & {\scriptsize{}$t=4$} & {\scriptsize{}$t=5$} & {\scriptsize{}$t=6$} & {\scriptsize{}$t=7$}\tabularnewline
\hline
\hline
{\scriptsize{}-} & {\scriptsize{}-} & {\scriptsize{}-} & {\scriptsize{}-} & {\scriptsize{}Donation 5} & {\scriptsize{}Donation 6} & {\scriptsize{}Donation 7}\tabularnewline
\hline
{\scriptsize{}-} & {\scriptsize{}-} & {\scriptsize{}-} & {\scriptsize{}Donation 4} & {\scriptsize{}Donation 4} & {\scriptsize{}Donation 5} & {\scriptsize{}Donation 6}\tabularnewline
\hline
{\scriptsize{}-} & {\scriptsize{}-} & \textbf{\scriptsize{}Donation 3} & \textbf{\scriptsize{}Donation 3} & \textbf{\scriptsize{}Donation 3} & {\scriptsize{}Donation 4} & {\scriptsize{}Donation 5}\tabularnewline
\hline
{\scriptsize{}-} & {\scriptsize{}Donation 2} & {\scriptsize{}Donation 2} & {\scriptsize{}Donation 2} & {\scriptsize{}Donation 2} & \textbf{\scriptsize{}Donation 3} & {\scriptsize{}Donation 4}\tabularnewline
\hline
{\scriptsize{}Donation 1} & {\scriptsize{}Donation 1} & {\scriptsize{}Donation 1} & {\scriptsize{}Donation 1} & {\scriptsize{}Donation 1} & {\scriptsize{}Donation 2} & \textbf{\scriptsize{}Donation 3}\tabularnewline
\hline
\end{tabular}{\scriptsize{}}\\
{\scriptsize \par}

The Table reveals that our intervention (Donation $3$) may differentially
impact the first subsequent donor relative to the second, third, and
later subsequent donors. Hence, there is virtue in estimating the
econometric model in equation \ref{eq:donor_i} separately for first
donations following ours, the second donation, etc.{*}\footnote{DR: I'm not sure what you are proposing here. If we run a separate
regression for each, we will lose some power.} By using two or more observations per page, i.e., both the first
and second donations after us, we can include page fixed effects.
Another way to estimate heterogeneous treatments effects according
to the subsequent donations order is to interact the treatment dummy
with the order of the donation.{*}\footnote{DR: A simple interaction might be a reasonable restricted model, but
we could probably do something more sophisticated?} But when using the the order of the donation after our random intervention
\textendash{} as above \textendash{} the following issue emerges:
Although for the treated pages it is clear who donates directly after
us, the timing does not coincide with donations in the control-pages.
Therefore, we need to construct a rule on the order of donation after
us for the control-group pages.{*}\footnote{DR: Perhaps our model could examine 'size of the k'th donation', where
the count k excludes our own donation.

Alternatively, all donations could be in the same regression, with
flexible controls for 'time elapsed since page founding' and perhaps
dummies for 'count of previous donations excluding our treatment.'

At the same time, we could differentiate the treatment effect by 'time
since treatment' and 'donation order since treatment'; I think this
should all be identifiable.}

Peer effects in charitable giving might exist in the control group,
as studied in \citet{Smithetal2013}. The authors base identification
of peer effects on ``\textit{plausibly} exogenous variation in the
observed history of donations''. Through our field experiment we
can base identification on \textit{definitely} exogenous variation,
as we will induce this variation ourselves (via our treatments). Using
only observations from the treated pages, we can estimate the linear-in-means
econometric model of \citet{Smithetal2013}:{*}\footnote{DR: This needs further clarification.

1. If we merely estimate this model on the treated pages, including
all donations, this will combine exogenous and endogenous variation,
thus making us vulnerable to similar concerns as in Smith et al. Should
we perhaps do something like 'instrumenting' for the average prior
donation using our treatment?

2. Should this be the mean of donations viewed on the page, or the
overall mean (or both, to test the distinction)? ... maybe other specifications
too?}

\begin{equation}
y_{pi}=\alpha+\frac{\lambda}{n_{p}-1}\sum_{j\neq i}^{n_{p}}y_{jp}+\varepsilon_{pi},\label{eq:lim_i_smith}
\end{equation}
as well as the suggested treatment effect model regarding large/small
donations {[}define the variables here{]}:

\begin{equation}
y_{pi}=\alpha+T_{pi}\delta+z_{pi}\theta+\varepsilon_{pi},\label{eq:donor_i_smith}
\end{equation}
where $T_{pi}$ is our randomly assigned donation treatment.

Comparison, especially using model \eqref{eq:donor_i_smith}, allows
us to test the appropriateness of the exogeneity assumption without
a random and anonymous donation such as in our field experiment.{*}\footnote{DR: It would be great to formalise such a test, or series of tests.
Will we test the specific assumptions of their model?}

Instead of using information on donor $i$ to estimate the treatment
effect, we can set up an econometric model at the page level, $p=1,...,P$:

\begin{equation}
y_{p}=\alpha+I_{p}\beta+x_{p}\gamma+\varepsilon_{p},\label{eq:page_p}
\end{equation}
in which $y_{p}$ denotes some distributional feature of the page,
e.g., the mean amount donated (mode, min, max, etc.), or an indicator
variable taking value $1$ if the target was reached or the time needed
to reach the target.{*}\footnote{DR: This is attractive -{}- we were working on estimating this for
our blocking procedure anyway}

\section{More general notation for treatment assignment and design and scoping
\textendash{} moving this to .Rmd file}

Treatments: Small donation $(D=s)$, anonymous large donation $(D=l)$,
or no donation $(D=0)$. The large donation (``seedsize) will be
set equal to twice the mean of previous donation on this page at the
point we are making this donation, rounded to the nearest �10 increment,
while the small donation will be\emph{ half of the previous }mean,
rounded to the nearest �5.\footnote{This follows the focal analysis of Smith et al. However, future experiments
will be designed to determine the optimal seed size for maximizing
future donations, following techniques proposed by Kasy's ``Dynamic
experimental design for policy choice''.}


MOVED: discussion of treatment assignment
\pagebreak{}


\bibliographystyle{econometrica}
\phantomsection\addcontentsline{toc}{section}{\refname}\bibliography{bibliography}

\end{document}
