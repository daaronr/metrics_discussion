%% LyX 2.2.1 created this file.  For more info, see http://www.lyx.org/.
%% Do not edit unless you really know what you are doing.
\RequirePackage{fixltx2e}
\documentclass[12pt,english,marginparwidth=8cm, marginparsep=3mm]{article}
\usepackage{lmodern}
\renewcommand{\sfdefault}{lmss}
\renewcommand{\ttdefault}{lmtt}
\usepackage[T1]{fontenc}
\usepackage[latin9]{inputenc}
\usepackage[a4paper]{geometry}
\geometry{verbose,tmargin=1.5cm,bmargin=1.5cm,lmargin=1.5cm,rmargin=1.5cm}
\pagestyle{plain}
\usepackage{color}
\usepackage{babel}
\usepackage{enumitem}
\usepackage{amsmath}
\usepackage{amsthm}
\usepackage{setspace}
\usepackage[authoryear]{natbib}
\PassOptionsToPackage{normalem}{ulem}
\usepackage{ulem}
\onehalfspacing
\usepackage[unicode=true,pdfusetitle,
 bookmarks=true,bookmarksnumbered=false,bookmarksopen=false,
 breaklinks=true,pdfborder={0 0 0},pdfborderstyle={},backref=false,colorlinks=true]
 {hyperref}
\usepackage{breakurl}

\makeatletter

%%%%%%%%%%%%%%%%%%%%%%%%%%%%%% LyX specific LaTeX commands.
%% Because html converters don't know tabularnewline
\providecommand{\tabularnewline}{\\}

%%%%%%%%%%%%%%%%%%%%%%%%%%%%%% Textclass specific LaTeX commands.
\numberwithin{equation}{section}
\numberwithin{figure}{section}
\numberwithin{table}{section}
\newlength{\lyxlabelwidth}      % auxiliary length 

\@ifundefined{date}{}{\date{}}
%%%%%%%%%%%%%%%%%%%%%%%%%%%%%% User specified LaTeX commands.
\usepackage{graphicx} 
\usepackage{hyperref}
  \hypersetup{colorlinks=true,citecolor=blue,linkcolor=blue}

%\setlength{\marginparwidth}{5cm}

\makeatother

\begin{document}

\subsection{Background}

We are running an experiment on peer/social influences on donations
in an online fundraising setting. The aim of this study is to assess
the impact larger-than-average (and smaller-than-average) additional
donation(s) to a fundraiser\textquoteright s page can have on subsequent
donations to that page. This is of academic and practical interest,
including for those in the Effective Altruism (EA) community who are
considering whether to donate independently or via friends and strangers'
fundraisers.

We are running a field experiment/randomized controlled trial in a
natural environment. Essentially, we observe all newly created fundraising
pages on the JustGiving platform meeting our criteria. We will assign
these to Treatment and Control groups in a way specified in our pre-registration
(blocked randomization, with the randomization component {[}possibly{]}
based on an ex-post observable device). For the Treatment pages we
will make a single anonymous donation on this page (from our personal
funds), with gift-aid, of a pre-specified size {[}relative to prior
contributions{]}. The unit of randomization is the fundraising page.
The 'participants' are all subsequent visitors to the fundraising
page, including the page founder (who may take subsequent actions
to promote the page).

We will continue to run the above until we have attained sufficient
statistical power (as defined in our preregistration) or until we
run out of funds (also defined). We will observe and collect data
from these sites until they are closed, or until the point where each
of these are likely to have obtained all or most of their donations
(also defined).

We will analyze the impact of our treatment on several outcomes, principally
focusing on it's impact on {*}total{*} contributions (excluding our
treatment contribution), but also analyzing the time-distribution
of these and the nature of the peer-effects, and examining heterogeneity
in was specified in our pre-analysis plan (PaP). 

\section{\label{sec:EMs}Econometric Models}

We exogenously assign whether we add an anonymous small donation $(D=s)$,
an anonymous large donation $(D=l)$, or no donation $(D=0)$ to a
page (the proposed exogenous assignment procedure is described in
section {[}ref{]} below). 

\emph{A simple model of a single treatment is proposed below, by Eirini
Tatsi.}

Suppose our sample consists of treated and control pages. 

We model the donation of subsequent donors, as well as the total amount
raised by the page. 

A standard model, following \citet{Bursztynetal2014}, for donor $i=1,...,n$
is: 

\begin{equation}
y_{i}=\alpha+I_{i}\beta+x_{i}\gamma+\varepsilon_{i},\label{eq:donor_i}
\end{equation}
in which $y$ denotes \textendash{} for instance \textendash{} the
amount donated by individual $i=1,...,n$; $\alpha$ is a constant
term, $I_{i}$ is an indicator variable taking the value $1$ if the
individual belongs to the treatment group and $0$ otherwise, $x$
are donors' characteristics while $\varepsilon_{i}$ is an independent
and identically distributed error term.{*}\footnote{DR: We may later allow for time and ordering effects... donations
may come in waves instigated by external unobservable factors.} The outcome variable $y$ is necessarily the amount the individual
donated or how much they donated compared to a benchmark amount because
we only observe donations and not the decision to donate (as in \citealp{Bursztynetal2014}).
Scalar parameter $\beta$ is the parameter of interest as it identifies
the one of the experimental treatment effects of interest{*}\footnote{DR: We need to define this treatment effect more carefully. Not sure
we can identify the impact 'conditional on a positive donation. Also
for policy purposes we are also interested in other outcomes, particularly
the total amount raised, which will be affected through several channels,
including the arrival rate, the rate of donating a positive amount
after arriving, and by indirect peer effects.} Matrix $x$ might include the donor's gender (if this is identifiable
using the names)\textendash{} or if the donor has a social-media connection
to the person who started the charity page{*}\footnote{DR: do we observe the social media connection?}
(since according to \citealp{Smithetal2013} those who donate are
usually friends, family or colleagues of the person who started the
charity page); it can also include an indicator variable taking value
$1$ if the donor was anonymous.{*}\footnote{DR: Careful-{}-the decision to be anonymous is likely to be affected
by the treatment here} 

In reality, donors to the same page might share common unobserved
characteristics, e.g., preferences to contribute to an animal-rights
cause or an enviromental cause. Therefore, the econometric model in
equation \ref{eq:donor_i} should include page $p=1,...,P$ fixed
effects:{*}\footnote{DR: Calling them 'fixed efects' may confuse people as these will be
exogenous to the treatments and we aren't planning to use the standard
\textquotedbl{}fixed effects\textquotedbl{} estimator.}

\begin{equation}
y_{i}=\alpha_{p}+I_{i}\beta+x_{i}\gamma+\varepsilon_{i}.\label{eq:donor_i_page_fe}
\end{equation}
For the same reason, in estimation standard errors could be clustered
at the page level. Writing the model in \ref{eq:donor_i_page_fe}
for the $n$ observations of the $P$ pages, we get:

\begin{equation}
Y_{n}=\alpha_{n}+I_{n}\beta+X_{n}\gamma+\varepsilon_{n},\label{eq:donor_n_page_fe}
\end{equation}
in which $Y_{n}$ denotes the $n\times1$ vector of outcomes, $\alpha_{n}=\left[\alpha_{1}\cdots\alpha_{1}\cdots\cdots\alpha_{P}\cdots\alpha_{P}\right]$,
$I_{n}$ denotes the $n\times1$ vector of treatment dummies, $X_{n}$
is the $n\times k$ matrix of explanatory variables and $\varepsilon_{n}$
is the $n\times1$ vector of i.i.d. error terms. To allow for errors
of of the same page to be correlated,\footnote{According to \citealp{Barriosetal2012}, if ``the covariate of interest
is randomly assigned at the cluster level'' \textendash{} herein
page \textendash{} calculating standard errors beyond clustering might
be redundant.} we can define:

\begin{equation}
Y_{n}=\alpha_{n}+I_{n}\beta+X_{n}\gamma+u_{n},\label{eq:donor_n_page_fe_sem}
\end{equation}

\begin{equation}
u_{n}=\rho W_{n}u_{n}+\varepsilon_{n},
\end{equation}
in which $W_{n}$ is a block-diagonal weights matrix with non-zero
elements if two donations were made at the same page. 

When visiting the charity webpage, one can see immediately the previous
five donations. Assuming for expositional purposes that our assigned
donation treatment is the \emph{third}, then the information visible
is as follows:\\

{\scriptsize{}}%
\begin{tabular}{|c|c|c|c|c|c|c|}
\hline 
{\scriptsize{}$t=1$} & {\scriptsize{}$t=2$} & {\scriptsize{}$t=3$} & {\scriptsize{}$t=4$} & {\scriptsize{}$t=5$} & {\scriptsize{}$t=6$} & {\scriptsize{}$t=7$}\tabularnewline
\hline 
\hline 
{\scriptsize{}-} & {\scriptsize{}-} & {\scriptsize{}-} & {\scriptsize{}-} & {\scriptsize{}Donation 5} & {\scriptsize{}Donation 6} & {\scriptsize{}Donation 7}\tabularnewline
\hline 
{\scriptsize{}-} & {\scriptsize{}-} & {\scriptsize{}-} & {\scriptsize{}Donation 4} & {\scriptsize{}Donation 4} & {\scriptsize{}Donation 5} & {\scriptsize{}Donation 6}\tabularnewline
\hline 
{\scriptsize{}-} & {\scriptsize{}-} & \textbf{\scriptsize{}Donation 3} & \textbf{\scriptsize{}Donation 3} & \textbf{\scriptsize{}Donation 3} & {\scriptsize{}Donation 4} & {\scriptsize{}Donation 5}\tabularnewline
\hline 
{\scriptsize{}-} & {\scriptsize{}Donation 2} & {\scriptsize{}Donation 2} & {\scriptsize{}Donation 2} & {\scriptsize{}Donation 2} & \textbf{\scriptsize{}Donation 3} & {\scriptsize{}Donation 4}\tabularnewline
\hline 
{\scriptsize{}Donation 1} & {\scriptsize{}Donation 1} & {\scriptsize{}Donation 1} & {\scriptsize{}Donation 1} & {\scriptsize{}Donation 1} & {\scriptsize{}Donation 2} & \textbf{\scriptsize{}Donation 3}\tabularnewline
\hline 
\end{tabular}{\scriptsize{}}\\
{\scriptsize \par}

The Table reveals that our intervention (Donation $3$) may differentially
impact the first subsequent donor relative to the second, third, and
later subsequent donors. Hence, there is virtue in estimating the
econometric model in equation \ref{eq:donor_i} separately for first
donations following ours, the second donation, etc.{*}\footnote{DR: I'm not sure what you are proposing here. If we run a separate
regression for each, we will lose some power.} By using two or more observations per page, i.e., both the first
and second donations after us, we can include page fixed effects.
Another way to estimate heterogeneous treatments effects according
to the subsequent donations order is to interact the treatment dummy
with the order of the donation.{*}\footnote{DR: A simple interaction might be a reasonable restricted model, but
we could probably do something more sophisticated?} But when using the the order of the donation after our random intervention
\textendash{} as above \textendash{} the following issue emerges:
Although for the treated pages it is clear who donates directly after
us, the timing does not coincide with donations in the control-pages.
Therefore, we need to construct a rule on the order of donation after
us for the control-group pages.{*}\footnote{DR: Perhaps our model could examine 'size of the k'th donation', where
the count k excludes our own donation. 

Alternatively, all donations could be in the same regression, with
flexible controls for 'time elapsed since page founding' and perhaps
dummies for 'count of previous donations excluding our treatment.' 

At the same time, we could differentiate the treatment effect by 'time
since treatment' and 'donation order since treatment'; I think this
should all be identifiable.}

Peer effects in charitable giving might exist in the control group,
as studied in \citet{Smithetal2013}. The authors base identification
of peer effects on ``\textit{plausibly} exogenous variation in the
observed history of donations''. Through our field experiment we
can base identification on \textit{definitely} exogenous variation,
as we will induce this variation ourselves (via our treatments). Using
only observations from the treated pages, we can estimate the linear-in-means
econometric model of \citet{Smithetal2013}:{*}\footnote{DR: This needs further clarification. 

1. If we merely estimate this model on the treated pages, including
all donations, this will combine exogenous and endogenous variation,
thus making us vulnerable to similar concerns as in Smith et al. Should
we perhaps do something like 'instrumenting' for the average prior
donation using our treatment?

2. Should this be the mean of donations viewed on the page, or the
overall mean (or both, to test the distinction)? ... maybe other specifications
too?}

\begin{equation}
y_{pi}=\alpha+\frac{\lambda}{n_{p}-1}\sum_{j\neq i}^{n_{p}}y_{jp}+\varepsilon_{pi},\label{eq:lim_i_smith}
\end{equation}
as well as the suggested treatment effect model regarding large/small
donations {[}define the variables here{]}:

\begin{equation}
y_{pi}=\alpha+T_{pi}\delta+z_{pi}\theta+\varepsilon_{pi},\label{eq:donor_i_smith}
\end{equation}
where $T_{pi}$ is our randomly assigned donation treatment. 

Comparison, especially using model \eqref{eq:donor_i_smith}, allows
us to test the appropriateness of the exogeneity assumption without
a random and anonymous donation such as in our field experiment.{*}\footnote{DR: It would be great to formalise such a test, or series of tests.
Will we test the specific assumptions of their model?}

Instead of using information on donor $i$ to estimate the treatment
effect, we can set up an econometric model at the page level, $p=1,...,P$:

\begin{equation}
y_{p}=\alpha+I_{p}\beta+x_{p}\gamma+\varepsilon_{p},\label{eq:page_p}
\end{equation}
in which $y_{p}$ denotes some distributional feature of the page,
e.g., the mean amount donated (mode, min, max, etc.), or an indicator
variable taking value $1$ if the target was reached or the time needed
to reach the target.{*}\footnote{DR: This is attractive -{}- we were working on estimating this for
our blocking procedure anyway} 

\section{More general notation for treatment assignment and design and scoping
\textendash{} moving this to .Rmd file}

Treatments: Small donation $(D=s)$, anonymous large donation $(D=l)$,
or no donation $(D=0)$. The large donation (``seedsize) will be
set equal to twice the mean of previous donation on this page at the
point we are making this donation, rounded to the nearest �10 increment,
while the small donation will be\emph{ half of the previous }mean,
rounded to the nearest �5.\footnote{This follows the focal analysis of Smith et al. However, future experiments
will be designed to determine the optimal seed size for maximizing
future donations, following techniques proposed by Kasy's ``Dynamic
experimental design for policy choice''.}

\subsection{Treatment assignment}

Our principal ``Main Hypotheses 1'' is to estimate each of the three
Average Treatment Effects on total subsequent donations, for SEED-H
and SEED-L relative to the control (no seed) and relative to each
other.\footnote{We also aim to estimate effects on individual subsequent donations
and other measures, as discussed in our pre-registration. } We have several constraints: our funding (a total of �5000), our
time (we need to implement this over a period of at most 4 weeks),
and the number of pages we expect to arise. Given this will assign
our treatments (both the \emph{share} \emph{of treatments/controls
}and the ``stratification''\emph{ }to minimize the expected mean
squared error (MSE) of the aforementioned estimates, weighting each
of the three equally.\emph{ (Is this equivalent to maximising the
power to detect an effect of a given size?)}. 

Our treatment assignment is influenced by Kasy (2016) and Barrios
(2014?). 

There are two important differences between our environment and those
discussed in these papers. 
\begin{enumerate}
\item In the previous papers, all baseline characteristics of the sample
are known in advance of the assignment process. In our case, we will
only observe these as new pages are generated over time. As each page
arises (and its initial characteristics are observed), we must decide
which treatment or control to assign. (However, we have a substantial
history of previous pages which we may use to infer the likely distribution
of future pages). 
\item The previous papers considered identically and independently distributed
potential outcomes, without clustering or bunching. In contrast, our
experiment will be implemented over a period of weeks or months. Outcomes
may be correlated during particular time periods, because of common
unobservable variables.\footnote{E.g., a news event may spur the creation of a particular series of
fundraising likely to attract particularly high/low donations.}\linebreak{}
\end{enumerate}
\uline{Kasy's setup: }

1. Sample observations $i$ from the population... collect baseline
covariates $X_{i}$. 

2. Using ... X and {[}possible randomization{]} ... allocate units
i to treatments $D_{i}$. 

3. {[}Post-experiment{]} collect ... outcomes $Y_{i}$ ... {[}use{]}
to calculate estimators $\hat{{\beta}}$ of the ... $\beta$ that
we are interested in {[}esp ATE{]}. 

4. ... Loss , $L(\hat{\beta},\beta)$ ... commonly $(\hat{\beta}-\beta)^{2}$.\linebreak{}

\uline{Our setup, written in Kasy's terminology}

0.1. Observe outcomes $Y_{it}$ (e.g., total amount raised) for a
``history'' of pages , where $t$ indexes the order of start date/time
of the page, $t<0$. Observe baseline (pre-hypothetical-treatment)
covariates $X_{it}$ including exact date/time of page foundation
$\tau$. Observe ``pseudo-treatment'' variables$\tilde{D_{it}}$(e.g.,
early donations before hypothetical treatment time). 

0.2. Observe or estimate parameters (e.g., $\hat{V}(\beta|X)$ based
on non-experimental data (our own analysis, Smith et-al)

1. Define rule for assigning treatments and controls using the above,
as a function of time and baseline covariates. (Should this rule dynamically
adjust to outcome variation?)

2. After officially starting experiment ($t=0$), treatable pages
arise at some rate $\rho(\tau)$, revealing baseline covariates $X_{it}$
including exact date/time of page foundation $\tau$.

3. As each page arises, assign it to the treatment(s)/control as a
function of baseline covariates (and a possible randomisation device,
and possibly previously collected measures: outcome variance by treatment,
etc). Implement this treatment. 

4. Continue experiment until pre-specified stopping rule (see below).
Collect $j$ outcome measures $Y_{ij}$ for each page.

5. Calculate estimators$\hat{{\beta}}$ of the ... $\beta$ that we
are interested in (\emph{which estimators?})

6. Loss , $L(\hat{\beta},\beta)$ ... defined as $(\hat{\beta}-\beta)^{2}$
(this cannot be computed from realized data, of course`)

\subsubsection{Proposed randomization procedure I (a Barrios spin-off)}

\emph{(If we adopt this procedure)} We will estimate a \emph{predictive}
(ridge regression) model of the total amount raised based on pre-experiment
``historical'' data, fitted on variables (``baseline covariates'')
that can be observed at the time the page is created and started and
before it is potentially treated. 

For each new page we will use this model to predict the expected total
revenue, categorizing the page by ``quantile of the distribution
of similar previous pages'' (again from pre-experiment data, but
perhaps adjusting as pages come in?). Within each predicted quantile,
we will essentially alternate between treatment and control (select
from an urn with two balls without replacement). The initial treatment
for all bins will be chosen randomly, according to a pre-registered
public randomisation device.

We will choose the size of the aforementioned quantiles (e.g., quartiles
or deciles) to trade off between efficiency with respect to balancing
the size of predicted donation between treatment and control, and
to achieve a balance between treatment and control within small time
intervals. We can calibrate this using our ``historical'' data.
using Monte-carlo simulations to determine which procedures consistently
yields the lowest mean-squared error. (In doing this we might want
to separate a fitting and a testing sample to avoid overfitting? \footnote{I suspect that a succession of treatments or controls in the same
time period may make our estimates vulnerable to unobservable time-specific
shocks; as this would be hard to diagnose using standard techniques
and simulations without a very long history. Intuitively, we would
like an approximate balance of control and treatment observations
within each 4 hours. However, let us explore whether we can find a
more systematic approach.}

In doing the above we should also consider that the share of pages
in each ``expected revenue'' quantile as predicted from \emph{previous
data} will not be equal, as the baseline variables will change over
time; we should consider trends, and seasonality, and unobservable
time-specific shocks to this.

\subsubsection{Proposed randomization procedure II (Max Kasy to propose, presumably
with a Bayesian framework )}

\textbf{Pre-determined variables}

\subsection{Sample size and stopping rule}

Following Sagarin's ``p\_crit'' approach,\footnote{DR: This approach seems fairly intuitive and feasible/realistic. The
idea that (e.g.,) \textquotedblleft we will collect in batches of
150, and we will not stop until we attain p>p\_crit or (for futility)
whenever p>.5\textquotedbl{} seems like something that could be a
credible commitment, if preregistered. At the point of submitting
the paper with this plan preregistered, one could check whether they
indeed did collect data in these batches (assuming they can\textquoteright t
hide data) and could check whether they followed this plan. } we can continue to collect in batches, with prespecified stopping
rules, stopping either when
\begin{enumerate}
\item We find a treatment-control differential that is statistically significant
above some nominal p-value ``p\_crit''
\item For ``futility'', after the p-value exceeds some value. 
\end{enumerate}
The precise value of p\_crit can be calibrated so that the procedure
above yields less than an (e.g.) 5\% rate of false-positive. 

However, I would like to adapt this to more useful stopping ``targets''
involving achieving reasonable confidence intervals. 

We want to run the experiment until we have sufficient statistical
power both to assess the Smith et al results and to provide evidence
on whether giving on social fundraising pages is an effective strategy
for those in the EA movement. We aim to continue collecting data until
we attain \emph{both} of the below:
\begin{itemize}
\item Considering Smith et al's result that an increase in the mean contribution
by one unit increase subsequent contributions by 1/4 of a unit...
For the impact on \emph{donations while this donation is visible on
the page}, we will collect data until the 95\% CI of our estimate
(of the impact of a change in the mean contribution on a later contribution)
either does \emph{not} include 0.25, or it includes 0.25 and has less
than a width below 0.25 (i.e., within {[}0.125, 0.375{]}). 
\item Considering our 'EA policy' implications... For the \emph{total subsequent
donation} we aim to gather enough data so that the 95\% confidence
interval on the impact of our treatment allows either
\begin{itemize}
\item If the effect appears null or negative, to rule out a (positive) effect
greater than 15\% of the size of our average (large) seed contribution 
\item to find a significant positive effect that is bounded within 50\%
of the size of our average (large) seed
\end{itemize}
\end{itemize}
Before the main experiment, we will re-evaluate this and re-report
our proposed sample size/stopping rule, as well as reporting our ex-ante
power estimates.\medskip{}

\uline{Details}

We have committed �5000 of our own funds for this purpose.

Note that we have collected \textquotedbl{}Historical Data\textquotedbl{}
since we created our R plugin close to the start of year-2018. Our
analysis of this data suggests that the average ``early'' donations
in this category were around �20, implying that our average ``Large''
seed size will be around �40. (And the ``small'' seed size will
typically be �10). This would imply our funds will yield an expected
sample size of {[}see ``scoping'' work. (We aimed at 150 treatment
pages and 150 control pages from a previous calculation.)

However, after these funds are exhausted, if we still have not attained
the power we seek, we will look for additional funding sources. It
is not reasonable for us as researchers to specify a stopping rule
based on cost\textendash{} realistically, the project will always
continue as long as funds can be raised (and it would be difficult
to verify that one 'would have stopped). 

...

\uline{Max}

``greedy algorithm''

Bayesian: posterior variance 

Stopping with a narrow CI\textendash{} with the CI based on the \emph{prior
}estimation

Thompson sampling

\pagebreak{}

\bibliographystyle{econometrica}
\phantomsection\addcontentsline{toc}{section}{\refname}\bibliography{bibliography}

\end{document}
