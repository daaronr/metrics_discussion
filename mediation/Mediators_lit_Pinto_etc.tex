%% LyX 2.2.1 created this file.  For more info, see http://www.lyx.org/.
%% Do not edit unless you really know what you are doing.
\documentclass[english]{article}
\usepackage[T1]{fontenc}
\usepackage[latin9]{inputenc}
\usepackage{amstext}
\usepackage{esint}
\PassOptionsToPackage{normalem}{ulem}
\usepackage{ulem}
\usepackage{babel}
\begin{document}

\title{Mediators, selection, Roy models: review focused on issues relevant
to Parey et al}

\maketitle
\textbf{\emph{DR initial thoughts:}}

Suppose we observe treatment $T$ (e.g., allowed to enter first-choice
institution and course), intermediate outcome $M$ (e.g., completion
of degree in first-choice course and institution), and final outcome
$Y$ (e.g., lifetime income.)\\

Treatment $T$ directly affects final outcome $Y$
\[
\ensuremath{T\rightarrow Y}
\]

$T$ also affects intermediate outcome $M$.

\[
T\rightarrow M
\]
Intermediate outcome also affects final outcome $Y$.
\[
M\ensuremath{\rightarrow}Y
\]

With exogenous variation in $T$ and $M$ (or identified instruments
for each of these), we should be able to estimate each of these three
relationships as functions.

With homogeneous (and in a simplest case linear) effects and separate
we can use the functions estimated above to compute the total (direct
plus indirect) effect of $T$ on $Y$. We could also compute the share
of this effect that occurs \emph{via} the intermediate effect, i.e.,
$T\text{\ensuremath{\rightarrow} }M\text{\ensuremath{\rightarrow} }Y$.
This should be merely the composition of these two functions, or,
in the linear case, the product of the slope coefficients.

However, there are two major challenges to this estimation.
\begin{enumerate}
\item We (may) have a valid instrument for (exogenous variation in) $T$
only, and $M$ may arise through a process involving selection on
unobserved variables.
\item Each of the three above relationships (as well as the selection equation)
may involve heterogeneous functions; i.e., differential treatment
effects.
\end{enumerate}

\section{Econometric Mediation Analyses (Heckman and Pinto)}

\textbf{Econometric Mediation Analyses: Identifying the Sources of
Treatment Effects from Experimentally Estimated Production Technologies
with Unmeasured and Mismeasured Inputs }

\subsection*{Relevance to Parey et al}

We have an instrument for admission to one's first-choice institution
(and course). Our result show an impact of this admission on future
income, for at least some groups. However, this effect could come
through any of a number of channels. We observe some of these 'intermediate
outcomes', including course enrollment, course completion, medical
specialization, and location of residence, but we do not have specific
instruments for each of these (a lot of work might yield an instrument
for specialization; I hear there is a lottery).

\subsection{Summary and key modeling}

There is a 'production function' (cf income as a function of human
capital, opportunities, etc). Treatments (e.g., RCTs) may affect outcomes
through the following channels:
\begin{enumerate}
\item observable or proxied inputs (cf degree obtained, specialization entered,
years of study, moving away from parents, location of residence as
proxy for job opportunities)
\item unobservable/unmeasured inputs (cf human capital, social connections...)
\item the production function itself, the 'map between inputs and outputs
for treatment group members' (cf does the institution itself directly
shift the income?, does it change the impact of entering a specialization,
does human capital 'matter more' at some institutions?)
\end{enumerate}
If treatments affect unmeasured inputs in a way not statistically
independent of measured inputs, this biases estimates of the effect
of measured inputs. ``RCTs unaided by additional assumptions do not
allow the analyst to identify the causal effect of increases in measured
inputs on outputs ... {[}nor distinguish effects from changes in production
functions{]}.''

Here `` we can test some of the strong assumptions implicitly invoked''.

``Direct effects'' as commonly stated refer to the impact of both
channels 2 and 3 above. {[}DR: Channel 2 isn't really a direct effect
imho{]}\\

\textbf{Standard potential outcomes framework:}

\[
Y=DY_{1}+(1-D)Y_{0}
\]

\[
ATE=E(Y_{1}-Y_{0})
\]

\textbf{Production function
\[
Y_{d}=f_{d}(\mathbf{\mathbf{{\theta}}}_{d}^{p},\mathbf{\mathbf{{\theta}}}_{d}^{u},\mathbf{{X}}),d\in\left\{ 0,1\right\}
\]
}

... the function under treatment $d$; of proxied and unobserved inputs
that occur under state $d$, and baseline variables.\\

The production function implies:

\[
ATE=E\Big(f_{1}(\mathbf{\mathbf{{\theta}}}_{1}^{p},\mathbf{\mathbf{{\theta}}}_{1}^{u},\mathbf{{X}})-f_{0}(\mathbf{\mathbf{{\theta}}}_{0}^{p},\mathbf{\mathbf{{\theta}}}_{0}^{u},\mathbf{{X}})
\]
\\

We also consider counterfactual outputs, fixing treatment status and
proxied inputs:
\[
Y_{d,\bar{\theta_{d}}^{p}}=f_{d}(\mathbf{\mathbf{{\bar{{\theta}}}}}_{d}^{p},\mathbf{\mathbf{{\theta}}}_{d}^{u},\mathbf{{X}}),d\in\left\{ 0,1\right\}
\]
\\
This allows us to decompose ('as in the mediation literature'\emph{):}

\emph{ATE(d)=IE(d)+DE(d)}
\begin{itemize}
\item \emph{IE, Indirect} \emph{effect}: allows only the proxied inputs
to vary with the treatment (holds the rest fixed at one of the two
treatment statuses)
\item \emph{DE, Direct} \emph{effect}: allows technology and the distribution
of unobservables to vary with the treatment (holds proxied inputs
fixed at one of the two treatment statuses)\\
\end{itemize}
HP further decompose the direct effect into:
\begin{itemize}
\item $DE'(d,d')$: The impact of letting the treatment vary the map only
(fixing the rest at one of the two appropriate values)
\item $DE''(d,d')$: The impact of letting the treatment vary the unmeasured
inputs only (fixing the rest at one of the two appropriate values)
\end{itemize}
They use this to give two further ways of decomposing the ATE. \linebreak{}
\\
\textbf{Common assumptions}\\

``The \uline{standard literature} on mediation analysis in psychology
regresses outputs on mediator inputs'' ... often adopts the strongs
assumption of:
\begin{enumerate}
\item no variation in unmeasured inputs conditional on the treatment (implying
the effects of these are summarized by a treatment dummy) and\footnote{Cf 'winning institution' impacts human capital, social networks, etc
identically for everyone; e.g., not a greater effect for men then
for women, nor a greater effect for those entering particular specializations. }
\item full invariance of the production function: $f_{1}=f_{0}$.
\end{enumerate}
... which implies $Y_{d}=f(\mathbf{\theta}_{d}^{p},d,\mathbf{X})$.\\

\uline{Sequential ignorability (Imai et al, 10, '11)}:  Essentially,
independent randomization of both treatment status and measured inputs.\footnote{Cf 'winning institution' does not effect the specialization entered
nor the location of residence, nor are both determined by a third
factor.}\\

\emph{This sentence is hard to follow:}

``In other words, input $\theta_{d'}^{p}$ is statistically independent
of potential outputs when treatment is fixed at $D=d$ and measured
inputs are fixed at $\bar{\theta_{d'}^{p}}$ conditional on treatment
assignment $D$ and same preprogram characteristics $X$.''



This assumption yields the 'mediation formulas':

\begin{eqnarray*}
E(IE(d)|X)= & \int E(Y|\theta^{p}=t,D=d,X)\underbrace{\Big(dF_{(\theta^{p}|D=1,X)}(t)-dF_{(\theta^{p}|D=1,X)}(t)\Big)}_{{\text{Difference in distribution of proxy inputs}}} & (9)\\
E(DE(d)|X)= & \int\underbrace{\Big(E(Y|\theta^{p}=t,D=1,X)-E(Y|\theta^{p}=t,D=0,X)\Big)}_{\text{Dfc in expectations (unobservables, function) between treatments given proxy inputs }}expe\underbrace{{dF_{(\theta^{p}|D=1,X)}(t)}}_{\text{Distn proxy inputs for D=1}} & (10)
\end{eqnarray*}

\emph{(??F is presumably the distribution over the observables; where
did the unobservables go? They are in the expectations, I guess.)}\\
\emph{}\linebreak{}
\uline{Difference from RCT}

\emph{What RCT doesn't do:}
\begin{quotation}
{[}sequential ignorability{]} translates into ... \uline{no confounding
effects} on both treatments and measured inputs ... does not follow
from a randomized assignment of treatment ...{[}which{]} ensures independence
between treatment status and counterfactual inputs/outputs ... {[}but
\emph{not}{]} between proxied inputs $\theta_{d}^{p}$ and unmeasured
inputs $\theta_{d}^{u}$. {[}Thus \emph{not} between counterfactual
outputs and measured inputs is assumed in condition (ii).{]}
\end{quotation}
Cf, randomizing 'win first-choice institution' does not guarantee
that the choice (potential choice under winning/losing institution)
to enter a particular specialty is independent of (potential after
winning/losing institution) unobserved human capital gains at an institution.
The (potential) choiceof specialty is alsonot guaranteed choice independent
of potential incomes (holding proxy inputs like specialty constant)
if winning/losing institution.\linebreak{}

\emph{What RCT }\emph{\uline{does}}\emph{ do:}

RCT ensures ``independence between treatment status and counterfactual
inputs/outputs'', thus identifying 'treatment effects for proxied
inputs and for outputs.

CF, we can identify the impact of the treatment 'win first chosen
institution' on proxied input like 'enters a specialization' and on
outputs like 'income in observed years.'

\section{Pinto (2015), Selection Bias in a Controlled Experiment: The Case
of Moving to Opportunity}

\subsection*{Summary}
\begin{itemize}
\item ... 4000+ families targeted, incentive to relocate from projects to
better neighbourhoods.
\item Easy to identify impact of vouchers
\item Challenge (here) is to assess impact of \emph{neighborhoods} on outcomes.
\item Method here to decompose the TEOT into unambiguously interpreted effects.
Method applicable to 'unordered choice models with categorical instrumental
variables and multiple treatments'
\item Finds significant causal effect on labour market outcomes
\end{itemize}

\subsection*{Relevance to Parey et al}
\begin{enumerate}
\item We also have an instrument (DUO lottery numbers) cleanly identifying
the effect of the 'opportunity to do something' (in our case, to enter
the course at your preferred institution). However, we also want to
measure the impact of choices 'encouraged' by the instrument, such
as (i) attending the first choice course and institution and (ii)
completing this course. We also deal with unordered choices (i. enter
course and institution, enter course at other institution, enter other
course at institution, enter neither) (ii. choice of medical specialisation)
\item The geographic outcome is relevant to our second paper (impact on
'lives close to home')
\end{enumerate}

\subsection*{Introduction }

The causal link between neighborhood characteristics and resident's
outcomes has seldom been assessed.

\textbf{Treatments:}
\begin{itemize}
\item Control (no voucher)
\item Experimental: could use voucher to lease in low-poverty neighborhood
\item Section 8: Could use voucher in any () neighborhood
\end{itemize}
\emph{Many papers evaluate the ITT or TOT effects of MTO.}
\begin{itemize}
\item ITT: effect of being \emph{offered} voucher
\begin{itemize}
\item estimated as difference in average outcome of experimental vs control
families
\end{itemize}
\item TOT: effect for 'voucher compliers' (assuming no effect of simply
being \emph{offered} voucher on those who don't use it)
\begin{itemize}
\item estimated as ITT/compliance rate
\end{itemize}
\end{itemize}
\begin{quote}
{[}ITT and TOT{]} are the most useful parameters to investigate the
effects of \emph{offering} {[}EA{]} rent subsidising vouchers to families.
\end{quote}

\subsection*{Identification strategy brief}
\begin{itemize}
\item Vouchers as IVs for choice among 3 neighborhood alternatives (no relocation,
relocate bad, relocate good) \emph{{[}Cf: enter course and fp-institution,
enter course at other institution, do not enter course{]}}
\item Neighborhood causal effects as difference in counterfactual outcomes
among 3 categories
\item Challenge: ``MTO vouchers are insufficient to identify the expected
outcomes for all possible counterfactual relocation decisions''
\begin{itemize}
\item ... ``compliance with the terms of the program was highly selective
{[}Clampet-Lundquist and M, 08{]}''
\end{itemize}
\item Solution: Uses theory and 'tools of causal inference. Invokes SARP
to identify 'set of counterfactual relocation choices that are economically
justifiable'
\item \uline{Identifying assumption}: ``the overall quality of the neighborhood
is not directly caused by the unobserved family variables even though
neighborhood quality correlates with these unobserved family variables
due to network sorting''
\item 'Partition sample ... into unobserved subsets associated with economically
justified counterfactual relocation choices and estimate the causal
effect of neighborhood relocation conditioned on these partition sets.'
{[}\emph{what does this mean?{]}}
\end{itemize}

\subsection*{Results in brief}

``Relocating from housing projects to low poverty neighborhoods generates
statistically significant results on labor market outcomes ... 65\%
higher than the TOT effect for adult earnings.''

\subsection*{Framework: first for binary/binary (simplification)}

\textbf{First, for binary outcomes (simplified)}

$Z_{\omega}$: whether family $\omega$ receives a voucher \emph{(cf
institution-winning lottery number)}

$T_{\omega}$: whether family $\omega$ relocates (\emph{cf enters
first choice institution and course)}\\
\bigskip

\textbf{Counterfactuals}
\begin{itemize}
\item $T_{\omega}(z)$: relocation decision $\omega$ would choose if it
had been assigned voucher $z\in{0,1}$': vector of potential relocation
decisions (\emph{cf education choices) }for each voucher assignment
(\emph{cf lottery number)}
\begin{itemize}
\item Can partition into never-takers, compliers, always takers, and defiers
\end{itemize}
\item $(Y_{\omega}(0);Y_{\omega}(1$)): (Potential counterfactual) outcomes
(\emph{cf income, residence, etc}) when relocation decision is fixed
at 0 and 1, respectively
\end{itemize}
\bigskip

\textbf{Key ( standard) identification assumption: instrument independent
of counterfactual variables}

\[
(Y_{\omega}(0),Y_{\omega}(1),T_{\omega}(0),T_{\omega}(1))\perp Z_{\omega}
\]

\textbf{Standard result 1: ITT}

\begin{eqnarray*}
ITT=E(Y_{\omega}|Z_{\omega}=1)-(Y_{\omega}|Z_{\omega}=0)\\
=E(Y_{\omega}(1)-Y_{\omega}(0)|S_{\omega}=[0,1]')P(S_{\omega}=[0,1])+E(Y_{\omega}(1)-Y_{\omega}(0)|S_{\omega}=[1,0]')P(S_{\omega}=[0,1])
\end{eqnarray*}

i.e., ITT computation yields the sum of the 'causal effect for compliers'
and the 'causal effect for defiers, weighted by the probability of
each.

\bigskip

\textbf{Standard result 2: LATE}

\begin{eqnarray*}
LATE=\frac{{ITT}}{P(T_{\omega}=1|Z_{\omega}=1)-P(T_{\omega}=1|Z_{\omega}=0)}= &  & E(Y_{\omega}(1)-Y_{\omega}(0)|S_{\omega}=[0,1]')\\
if\:P(S_{\omega}=[0,1])=0
\end{eqnarray*}

i.e., the LATE, computed as the ITT divided by the 'first stage' impact
of the instrument, is the causal effect for compliers if there are
no defiers.

\subsection*{Framework for MTO \textendash{} multiple treatment groups, multiple
choices}
\begin{itemize}
\item $Z_{\omega}\in\{z_{1,}z_{2,}z_{3}\}$ for no voucher, experimental
voucher, and section 8 voucher, respectively
\item $T_{\omega}\in\{1,2,3\}$ ... no relocation, low poverty neighborhood
relocation, high poverty relocation
\item $T_{\omega}(z)$: relocation decision for family $\omega$ if assigned
voucher $z$
\end{itemize}
$\ensuremath{\rightarrow}$Response type for each family $\omega$
is that a three-dimensional vector: $S_{\omega}=[T_{\omega}(z_{1}),T_{\omega}(z_{2}),T_{\omega}(z_{3})]$.
\\

$\ensuremath{\rightarrow}$ \textbf{ITT }computation now measures
a weighted sum of effects across a subset of those response types
whose responses vary between the assignments being compared. \\
\\
\emph{Cf: }
\begin{itemize}
\item Considering the 'treatments': '1: enter other course at fp-inst, '2:
enter course at fp-inst', '3: enter course at non-fp inst'
\begin{itemize}
\item (I ignore other course at other institution for now)
\end{itemize}
\item Looking among those who won the course lottery (so we have a binary
instrument: wininst $Z_{\omega}\in{0,1\}}$
\item Our reduced-form estimates (regressions on the 'lottery number wins
institution' dummy) measures the probablility-weighted sum of:
\begin{itemize}
\item impact of institution within course ($T_{\omega}=$2 versus 3); for
those who would 'fully comply' (enter course at institution if $Z_{\omega}=1$,
enter course at other institution if 0)
\item impact of the course at fp-institution versus second-best course at
fp-institution for 'institution-loving' noncompliers; those who would
enter the course \emph{only }if they get the fp-institution and otherwise
another course at the same institution
\item effects for perverse defiers
\end{itemize}
\end{itemize}

\end{document}
