%% LyX 1.6.4 created this file.  For more info, see http://www.lyx.org/.
%% Do not edit unless you really know what you are doing.
\documentclass[english]{beamer}
\usepackage{mathptmx}
\usepackage[T1]{fontenc}
\usepackage[latin9]{inputenc}
\usepackage{textcomp}
\usepackage{amsmath}
\usepackage{graphicx}
\usepackage{amssymb}

\makeatletter

%%%%%%%%%%%%%%%%%%%%%%%%%%%%%% LyX specific LaTeX commands.
\newcommand{\lyxmathsym}[1]{\ifmmode\begingroup\def\b@ld{bold}
  \text{\ifx\math@version\b@ld\bfseries\fi#1}\endgroup\else#1\fi}

%% A simple dot to overcome graphicx limitations
\newcommand{\lyxdot}{.}


%%%%%%%%%%%%%%%%%%%%%%%%%%%%%% Textclass specific LaTeX commands.
 % this default might be overridden by plain title style
 \newcommand\makebeamertitle{\frame{\maketitle}}%
 \AtBeginDocument{
   \let\origtableofcontents=\tableofcontents
   \def\tableofcontents{\@ifnextchar[{\origtableofcontents}{\gobbletableofcontents}}
   \def\gobbletableofcontents#1{\origtableofcontents}
 }

%%%%%%%%%%%%%%%%%%%%%%%%%%%%%% User specified LaTeX commands.
\usetheme{Warsaw}
% or ...

\setbeamercovered{transparent}
% or whatever (possibly just delete it)

\makeatother

\usepackage{babel}

\begin{document}
\textbf{Notes:}


\lyxframeend{}\section{References :Friedman and Sunder/ Friedman and Cassar.}

Krzanowski, {}``Statistical Principles...'', 2007, Chapter 6 on
Anova, Chapter 8: association btwn varls, ch 9, exploring complex
data sets, ch 10, special topics{[}Gujarati, ch 13, also slightly
relevant {}``traditional econometric methodology ... specification'',
24.4 {}``tests of nonnested hypotheses''. Ch 15 on dummy variables,
15.13 -- dummies in semilog regressions, 16, dummy (or censored) dependent
variable -- 16.14 -- Tobit ... measure of the \% impact of a 1 std.
dev. change. 

\underbar{\bigskip}

Greene, p 7655: marginal effects in the censored regression model;
(also might estimate effect overall and effect conditional on positivity
separately)

\includegraphics{\string"C:/Documents and Settings/david reinstein/My Documents/My Dropbox/experiment_course_and_notes/statistics of experiments/pasted1\string".eps}

Test of Tobit specification (?): compare $\frac{1}{\sigma}\beta$
from Tobit model to the Probit estimate -- if the Tobit is specified
correctly, these are the same asymptotically.

Also Cragg/Greene test (? for correct specification of probability
$Y=0)$, tests for normality (Cheshire-Irish and Pagan-Vella)`, {}``conditional
moment tests'' (for omitted variables, error homoskedastic, error
normal)

\includegraphics{\string"C:/Documents and Settings/david reinstein/My Documents/My Dropbox/experiment_course_and_notes/statistics of experiments/twoeqnmodelpasted2\string".eps}

Fin and Schmidt (1984) (likelihood ratio) test for {}``restriction''
of Tobit model (that $\gamma$ in binary decision equation =$\frac{\beta}{\sigma}$
in regression equation).


\lyxframeend{}\section{Friedman and Sunder/ Friedman and Cassar}


\lyxframeend{}\subsection{Data analysis}

{}```Experimental data usually are new and in some respects unfamiliar,
so a descriptive summary is essential.\textquotedbl{} \ensuremath{\centerdot}

\textquotedbl{}interocular trauma test\textquotedbl{} (Savage, 1954)
\ensuremath{\centerdot} 

When graphs of the raw data (the first step) are convincing, \textquotedbl{}...
do we really need any formal statistical tests? The traditional answer
in most natural sciences is no ... a few economists ... agree, but
most economists will not be convinced without formal tests.\textquotedbl{} 

\underbar{\textquotedbl{}Experimental error\textquotedbl{}} 
\begin{itemize}
\item \textquotedbl{}To the extent that you get different results on replication
... your analysis must deal with experimental error.\textquotedbl{} 
\item \textquotedbl{}... two sources: measurement and sampling\textquotedbl{}
... try to build in redundancy to avoid the former \ensuremath{\centerdot} 
\item {}``Observed difference in means might be due to experimental error,
and we need statistical techniques to evaluate that possibility ...
the scale of experimental error is hard to know in advance.\textquotedbl{}
\end{itemize}
\underbar{Sampling error:}
\begin{itemize}
\item -\textquotedbl{}Arises to the extent that your sample is not representative
of the underlying population\textquotedbl{} 
\item \textquotedbl{}Consider the collection of all possible trial outcomes
given your treatments ... the population of outcomes\textquotedbl{} 

\begin{itemize}
\item DR: \textquotedbl{}Population\textquotedbl{} of individuals' responses
to the same treatment 
\end{itemize}
\item \textquotedbl{}There is always some variability in the population
because of uncontrolled nuisances such as subject's attention to the
task\textquotedbl{} 
\item \textquotedbl{}If you knew the population distribution, your inferential
task would be trivial.\textquotedbl{} 

\begin{itemize}
\item DR: Test hypotheses on the population mean, median, etc., using the
sample mean and variance \ensuremath{\centerdot} 
\end{itemize}
\end{itemize}
\underbar{Sampling issues} 
\begin{itemize}
\item Random sample versus stratified/balanced sample 
\item \textquotedbl{}A balanced sample will tend to produce smaller errors
than a random sample of the same size to the extent that the outcomes
differ across segments, the segments are observable, and their weights
in the population are known.\textquotedbl{} \ensuremath{\centerdot} 

\begin{itemize}
\item Balancing: For each treatment, try to draw observations proportionally
to the weight of various segments of the population.
\end{itemize}
\item DR: A relevant example:

\begin{itemize}
\item \textquotedbl{}...Suppose an experimenter wants to measure the degree
of altruism in individual subjects. If he selects subjects in the
usual way, advertising the opportunity to earn `substantial cash rewards'
in undergraduate economics classes and signing up volunteers, his
altruism measures probably will not be typical of the population of
U.S. residents.\textquotedbl{}
\end{itemize}
\item ... There is selection bias ... yielding an \textquotedbl{}unbalanced,
non-random biased sample.\textquotedbl{} \ensuremath{\centerdot}
\end{itemize}
\underbar{Experimental data's advantages}; 
\begin{itemize}
\item Problems with happenstance samples are also serious 
\item With good experimental design, can \textquotedbl{}ensure that the
focus variables vary independently and over a sufficient range\textquotedbl{} 
\item DR: Can get independent variation in X; E.g., $f(\epsilon_{i}|X_{i})=f(\epsilon_{i})\rightarrow E(\epsilon_{i}|X_{i})=E(\epsilon_{i})=0$
and $E(f(\epsilon_{i})|X_{i})=E(f(\epsilon_{i}))$
\end{itemize}
\underbar{But experimental data vulnerable}\emph{:}

\textquotedbl{}the trials in a single experimental session are not
independent\textquotedbl{}:

\emph{What is an {[}independent{]} trial?}
\begin{itemize}
\item A single decision?

\begin{itemize}
\item DR: This is probably ok for non-interactive, random-realized-stage
individual decisions, although there may be `hyesteresis' ... for
paired stages the pair may be the trial
\end{itemize}
\item \ensuremath{\centerdot} The average group action each period? \ensuremath{\centerdot}
The run average?
\end{itemize}
Group effects: ...

\textquotedbl{}Our position (probably the more common view) is that
each choice is a trial, possibly interdependent with other trials
... Interdependence is taken into account ... Adjust the sample size
in light of the observed correlation {[}e.g., between pairs{]}.\textquotedbl{}
\textquotedbl{}For checking the statistical significance of a treatment
{[}to some interactive environment{]}, it is best to be conservative
and do the tests on run averages...

\bigskip{}


\underbar{Formal tests }

\underbar{I. Parametric:}
\begin{itemize}
\item \ensuremath{\centerdot} Normal distribution 

\begin{itemize}
\item Z-test or t-test whether true mean differs from null-hypothesis 
\item Pooled t-stat, 
\item or two-sample t-stat (with unequal variances): compare means of two
populations 
\item Paired/matched t-stat (on 1st difference) 
\end{itemize}
\end{itemize}
\underbar{Nonparametric:}
\begin{itemize}
\item Binomial test: for binary outcome 

\begin{itemize}
\item ... e.g., test whether agents choose NE with probability > 2/3 \ensuremath{\centerdot}
\end{itemize}
\item Signs test:

\begin{itemize}
\item Binomial with $p\lyxmathsym{\textasteriskcentered}=(\frac{1}{2})$
applied to matched pairs. {[}??{]} .. 
\item Can try to reject the null hypothesis that positive and negative differences
are equally likely, or whether (e.g.) $P(A>B)>.5$.
\item Note that with AB design \textquotedbl{}we do not have precisely matched
pairs because one observation in each pair is `before' and the other
is `after'.\textquotedbl{} 
\end{itemize}
\item \ensuremath{\centerdot} {[}parametric?{]} Chi-squared test: For comparing
binary outcomes without `panel' element \ensuremath{\centerdot} 
\item Fisher's exact test: 

\begin{itemize}
\item Binary outcomes, 2 treatments, assume row and column totals are pre-determined. 
\item DR: E.g., games with 1 winner and 1 loser?
\end{itemize}
\item Mann-Whitney test: Uses ranks in continuous-outcome data. Which sample
is `higher-ranked'? 

\begin{itemize}
\item \textquotedbl{}The test statistic is the sum of the ranks assigned
to those values from the first {[}e.g., treated{]} population.\textquotedbl{} 
\item $H_{0}:Pr[rank(X_{i})>$$rank(Y_{i})]=.5$ where $X_{i},Y_{i}$ are
random draws from (treated, untreated) populations??
\end{itemize}
\item Kruskal-Wallis: 

\begin{itemize}
\item Extension of Mann-Whitney to more than two samples 
\item DR: what are H0 and HA? All samples equally ranked versus some difference?
\end{itemize}
\item Wilcoxon signed ranks test: 

\begin{itemize}
\item Like signs test, but also considers (relative?) magnitude of $D_{i}=Y_{i}-X_{i}$.
{[}??{]}
\item Test-statistic is the sum of the ranks of the observations (of differences?)
that originally (?) had a positive sign. (?)
\item Tests $H_{0}:E(D)=0$. ??

\begin{itemize}
\item DR: So why should the magnitude matter to this? How does the matched
pairing help us? \ensuremath{\centerdot}
\end{itemize}
\end{itemize}
\item Bootstrap: 

\begin{itemize}
\item Resample from the data, generate estimator each time, look at mean
and standard deviation of this. 
\item |-Could do the same for matched differences or any other function
of the resampled data. ?:\textquotedbl{}Bias-corrected estimates\textquotedbl{}?
\end{itemize}
\end{itemize}
\textbf{Hoffman,Mccabe and Smith:}

\includegraphics{\string"C:/Documents and Settings/david reinstein/My Documents/My Dropbox/Givingexperiments/BeforeandAfterDesign/pasted2\string".eps}

\includegraphics{\string"C:/Documents and Settings/david reinstein/My Documents/My Dropbox/Givingexperiments/BeforeandAfterDesign/pasted1\string".eps}

Logit and OLS with treatment dummies


\lyxframeend{}\subsection{Reporting your Results}

\underbar{Coverage} 

\textquotedbl{}vary treatments and replicate sufficiently to obtain
a reasonably broad base of valid data ... select the most relevant
portion of the data for closer analysis {[}as long as{]} ... selection
does not distort the conclusions ... Describe your selection processes
{[}in your report {]}.\textquotedbl{}

\underbar{Econometrica guidelines:}
\begin{itemize}
\item \ensuremath{\centerdot} Include a lengthy Appendix, later replaced
by a 3-page version for publication \ensuremath{\centerdot}
\item Include instructions, \ensuremath{\centerdot} Hard copy of computer
screens if possible \ensuremath{\centerdot}
\item Main body: include section on experimental procedures \ensuremath{\centerdot} 
\item Make data available to referees, perhaps in article itself\bigskip{}

\end{itemize}
\textbf{Krzanowski, {}``Statistical Principles...'', 2007}

On ANOVA...

{}``Fisher's exact test'' (for difference in means between groups)
is nonparametric, makes no distributional assumptions -- it just permutes
the observations between groups and sees what changes.  \bigskip{}


\underbar{Designed experiments} : Principles of randomisation and
replication.

Note: Fisher's {}``plots'' = observations 

Randomised block design: Divides the field into strips or {}``blocks''
that run north to south, apply each treatment to the same number of
plots in each block. {[}\emph{Should we ever see blocks as treatments?{]}
... }Enables us to test for both block effects and treatment effects.
\emph{?how to do this nonparametrically?}

Latin square design: equal representation of all treatments in each
north-south strip, and also in each east-west strip. 

Balanced incomplete block design: every \emph{pair} of treatments
(?e.g., AB, AC, BC) occurs the same number of times across all blocks
(?) 

Lattice design if many treatments 

Split-plot design: To introduce additional treatments after the experiment
has started

\bigskip{}


{}``Factors'': Chemicals used in the field; {}``levels'': amount
of each chemical in the mixture

Factorial design: full set of possible combinations (different levels)
of each factor are represented; for a single {}``replicate'' --
allows estimation of all possible interaction effects. Can partition
into the main effects of each factor (averaged over the level of all
other factors) and the interaction effects (between pairs, triples,
etc). 

Fractional replication: Reduced design; loses some information on
interactions.

\bigskip{}


{}``The analysis of variance for \emph{any }experimental set-up can
recovered from that of regression analysis by approprate formulation
of a linear model.''

{}``... if we want to test the significance of a particular set of
variables then we need to find the 'extra sum of squares' due to this
set \emph{after }fitting all the other variables first. Since the
design is balanced the \emph{order} in which we fit the variables
is immaterial. 

{}``Once the design balance is lost ... the simple ANOVA formulation
is no longer appropriate.''

\bigskip{}
\underbar{Association between Variables}

In multivariate analysis no single variable has special status. 


\lyxframeend{}\section{Error structure, dependence, etc.}

Hey paper,

Loomes paper,

Gllamm estimator in stata, multilevel modeling (clustering?)

Arellano-Bond (?)

Discussion on ESA listserve
\end{document}
